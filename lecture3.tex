\documentclass[12pt]{article}
\usepackage[english]{babel}
\usepackage[utf8x]{inputenc}
\usepackage[T1]{fontenc}
\usepackage{scribe}
\usepackage{listings}

\Scribe{Akshay, Alan, Lakshya, Virendra}
\Lecturer{Abir De}
\LectureNumber{4}
\LectureDate{11 August 2022}
\LectureTitle{ML Terminology}

\lstset{style=mystyle}

\begin{document}
	\MakeScribeTop

%#############################################################
%#############################################################
%#############################################################
%#############################################################

%Intro
\noindent Over the last two lectures, we reviewed probability and linear algebra. Now, we introduce some basic ML terminology.

\section{An Example}

This example is in continuation from Lecture 2. We have a graph $G = (V,E)$, where $V=\{1,\ldots,n\}$ is the set of vertices, and $E$ is the set of edges. The degree of node $i$ is given by $d_i$. Define the value $x_i(t)$ to be the `opinion' of node $i$ at time $t$. ${x_1(0),\dots,x_n(0)}$ are given. For $t\ge 0$,
\begin{equation} \label{eq:1}
	x_i(t+1) = \frac {\sum_{j\in \mathcal{N}(i)} x_j(t)}{|\mathcal{N}(i)|}
\end{equation}
where $\mathcal{N}(i)$ denotes the set of neighbors of node $i$. That is, the opinion of $i$ at time $t+1$ is the average of the opinions of its neighbors at time $t$.
\\
\noindent Consider $\x(t) = [x_1(t) \cdots x_n(t)]^T$. Define $\A \in \mathbb{R}^{n\times n}$ such that $\A\x(t) = \x(t+1)$. From \eqref{eq:1}, $\A$ is a doubly-stochastic matrix, i.e., its entries are non-negative, and its rows and columns add up to $1$. For instance, the first row of $\A$ is of the form $[a_{1j}]^T$, where $a_{1j}=1/d_1$ if $j\in \mathcal{N}(i)$, else $0$.
\\

\noindent We state the following result from \cite{doublystochastic}. A stochastic matrix $\M$ is called semi-positive if all entries of some power $\M^\alpha$ are positive.
\begin{theorem}
	If $\M \in \mathbb{R}^{n\times n}$ is a semi-positive doubly stochastic matrix, then $\lim_{t\to \infty} \M^t = \frac{1}{n}\J$, where $\J \in \mathbb{R}^{n\times n}$ with all entries $1$.
\end{theorem}


\noindent Thus, if $\A$ is semi-positive and doubly stochastic, then for all nodes $i$, $\lim_{t\to \infty} x_i(t) = \frac{\sum_{j=1}^{n} x_j(0)}{n}$.
\\

\noindent Some points to note:
\begin{itemize}
	\itemsep0em
	\item $\A$ is symmetric
	\item $\A{\bf 1} = {\bf 1}$
	\item ${\bf 1^T}\A = {\bf 1^T}$
\end{itemize}


\section{Image Classification Problem}

We are given a set of images $\{I_1, \dots, I_n\}$, each corresponding to exactly one of three animal classes: cat, dog, or tiger. The pixel data of $I_i$ is encoded into feature-matrix $X_i \in \mathbb{R}^{d\times d}$, and IDs of the three classes are $\{0,1,2\}$ respectively. \textit{Binary} classification problems involve only two labels, usually $\{0,1\}$.
\\

\noindent The goal is to correctly predict the label of any (unseen) image. We can't possibly write an algorithm for this prediction, since the model itself changes with the input space. Moreover, an algorithm with one to one mappings for given set $\{X_i\}$ will perform poorly on unseen images. This is because the model has been completely overfit to these images.
\\

\noindent Our approach is to write a program that can design an algorithm that can classify $X_i$s. But this program doesn't know what the labels represent or how to differentiate $X_i$s. The problem with just the set of $X_i$'s is ill-defined. Therefore we need to teach this program using some data i.e. a set of examples of $X_i$'s which are already mapped to ($y_i$'s). This brings the idea of a training set.

\section{Training and Validation Sets}


\subsection{Training Set}
For the classification problem, the training set consists of pairs $(X_i,y_i)$. For each image matrix $X_i$, a label $y_i$ has already been provided, and this label is assumed to be correct. Assuming binary classification, we need to find a function $H : \mathbb{R}^{d\times d} \to \{0,1\}$, such that $H(X_i)=y_i$. As mentioned above, this function must give correct results for unseen images too.

\subsection{Validation Set}
How do we know that the algorithm is ``ready'' for use? Relying only on training set performance isn't a good idea, since the model would have overfit on this data. Thus, we reserve a section of the training data as the validation set. After training the model on the train set, its performance is evaluated on the validation set. Since the latter is unseen during training, we get a more reliable estimate of the model's performance.

%%%%%%%%%%% If you don't have citations then comment the lines below:
%
\bibliographystyle{abbrv}           % if you need a bibliography
\bibliography{mybib}                % assuming yours is named mybib.bib


%%%%%%%%%%% end of doc
\end{document}
